\documentclass[12pt,fleqn]{article}
%\usepackage {psfig,epsfig} % para incluir figuras em PostScript
\usepackage{amsfonts,amsthm,amsopn}
\usepackage{graphicx}
% XeLatéX já suporta UTF-8 nativamente
\usepackage[utf8]{inputenc}
\usepackage[T1]{fontenc}
\usepackage[brazil]{babel}
\usepackage{geometry}
\usepackage[intlimits]{amsmath}
\usepackage{listings}
\usepackage{xcolor}
\usepackage{booktabs}
\usepackage{url}
\usepackage{float}
\usepackage{hyperref}

% Configuração para código
\lstset{
    basicstyle=\ttfamily\small,
    breaklines=true,
    frame=single,
    numbers=left,
    numberstyle=\tiny,
    keywordstyle=\color{blue},
    commentstyle=\color{green!60!black},
    stringstyle=\color{red},
    backgroundcolor=\color{gray!10},
    showstringspaces=false
}

%alguns macros
\newcommand{\R}{\ensuremath{\mathbb{R}}}
\newcommand{\Rn}{{\ensuremath{\mathbb{R}}}^{n}}
\newcommand{\Rm}{{\ensuremath{\mathbb{R}}}^{m}}
\newcommand{\Rmn}{{\ensuremath{\mathbb{R}}}^{{m}\times{n}}}
\newcommand{\contcaption}[1]{\vspace*{-0.6\baselineskip}\begin{center}#1\end{center}\vspace*{-0.6\baselineskip}}
%=======================================================================
% Dimensoes da pagina
\geometry{
    a4paper,
    margin=2.5cm,
    top=3cm,
    bottom=2.5cm,
    left=2.5cm,
    right=2.5cm
}
\renewcommand{\baselinestretch}{1.15} % espaçamento entre linhas
       
% ser indulgente no preenchimento das linhas
\sloppy
 
\begin{document}

\pagestyle{plain}
\setlength{\parindent}{0pt}
\setlength{\parskip}{6pt}  

\begin{center}
{\Large \textbf{Capítulo 7 - Servidor Firewall e Aplicações}}
\end{center}
\begin{center}
{\textit{Wesley Ferreira, Henrique Moura, Gabriel Ornelas, Mateus Silva - IFG Campus Formosa}}
\end{center}

\section{Ficha Técnica}  

\begin{table}[h]
\centering
\begin{tabular}{|p{3cm}|p{10cm}|}
	\hline
	\textbf{Serviço} & \textbf{Descrição}\\\hline 
	Método de Comunicação & Protocolo TCP/UDP, portas configuráveis (22, 80, 443, etc.) \\\hline
	Funções & Controle de tráfego de rede, filtragem de pacotes, NAT, balanceamento de carga\\\hline
	Pacote para Instalação & iptables (já incluído no kernel Linux), iptables-persistent\\\hline
	Script de controle & \texttt{systemctl status netfilter-persistent}, \texttt{netfilter-persistent save} \\\hline
\end{tabular}
\caption{Ficha Técnica do servidor netfilter/iptables}
\end{table}

\section{Descrição do Servidor}

\subsection{Netfilter: O framework de firewall}

O \textbf{netfilter} é o framework de firewall do kernel Linux que permite controlar o tráfego de rede através de regras de filtragem de pacotes. É o sistema fundamental que implementa as funcionalidades de firewall no Linux, permitindo:

\begin{itemize}
    \item \textbf{Filtragem de pacotes}: controlar quais pacotes podem entrar ou sair do sistema
    \item \textbf{NAT (Network Address Translation)}: tradução de endereços de rede
    \item \textbf{Mangle}: modificação de cabeçalhos de pacotes
    \item \textbf{Logging}: registro de atividades de rede para auditoria
\end{itemize}

\subsection{iptables: A ferramenta de Gerenciamento}

O \textbf{iptables} é uma ferramenta de linha de comando utilizada para gerenciar o netfilter. É a interface mais comum para configurar as regras do firewall, permitindo:

\begin{itemize}
    \item criar, modificar e deletar regras de firewall
    \item definir políticas padrão para as cadeias
    \item configurar NAT e port forwarding
    \item implementar proteções contra ataques
\end{itemize}

O iptables trabalha com três tabelas principais:
\begin{itemize}
    \item \textbf{filter}: tabela padrão para filtragem de pacotes
    \item \textbf{nat}: para tradução de endereços de rede
    \item \textbf{mangle}: para modificação de pacotes
\end{itemize}

E cinco cadeias (chains) principais:
\begin{itemize}
    \item \textbf{INPUT}: pacotes destinados ao sistema local
    \item \textbf{OUTPUT}: pacotes originados do sistema local
    \item \textbf{FORWARD}: pacotes que passam pelo sistema (roteamento)
    \item \textbf{PREROUTING}: pacotes que chegam (NAT)
    \item \textbf{POSTROUTING}: pacotes que saem (NAT)
\end{itemize}

\section{iptables-legacy vs. nftables: A transição no Debian}

\subsection{Contexto Histórico}

Desde o Debian 10 ("Buster"), o framework padrão para firewall no Linux é o \textbf{nftables}, que substitui o iptables como ferramenta nativa. No Debian 12, quando você executa comandos \texttt{iptables}, está na verdade usando uma camada de compatibilidade (\texttt{iptables-nft}) que traduz a sintaxe do iptables para regras de nftables.

\subsection{Verificando qual versão está em uso}

\begin{lstlisting}[language=bash, caption=Verificação da versão do iptables]
# verificar qual versão do iptables está ativa
sudo update-alternatives --config iptables

# verificar se está usando nftables ou legacy
sudo iptables --version

# verificar as regras nativas do nftables
sudo nft list ruleset
\end{lstlisting}

\subsection{Implicações práticas}

\begin{itemize}
    \item \textbf{Compatibilidade}: os comandos iptables continuam funcionando normalmente
    \item \textbf{Performance}: nftables oferece melhor performance e flexibilidade
    \item \textbf{Debugging}: as regras podem aparecer diferentes quando visualizadas com \texttt{nft list ruleset}
    \item \textbf{Futuro}: recomenda-se aprender nftables para novos projetos
\end{itemize}

\subsection{Usando iptables-legacy (Opcional)}

Se você precisar usar o iptables "clássico" para compatibilidade com scripts antigos:

\begin{lstlisting}[language=bash, caption=Configuração do iptables-legacy]
# instalar iptables-legacy
sudo apt install iptables-legacy

# configurar para usar legacy
sudo update-alternatives --set iptables /usr/sbin/iptables-legacy
sudo update-alternatives --set ip6tables /usr/sbin/ip6tables-legacy
\end{lstlisting}

\section{Instalação}

O iptables já vem incluído no kernel Linux, mas para uma configuração completa, é necessário instalar alguns pacotes adicionais:

\begin{lstlisting}[language=bash, caption=Instalação do iptables]
# atualizar repositórios
sudo apt update

# instalar iptables e ferramentas relacionadas
sudo apt install iptables iptables-persistent netfilter-persistent

# verificar se o iptables está funcionando
sudo iptables -L -v
\end{lstlisting}

Para verificar se o módulo do kernel está carregado:

\begin{lstlisting}[language=bash, caption=Verificação de módulos]
# verificar módulos carregados
lsmod | grep iptable

# carregar módulos se necessário
sudo modprobe iptable_filter
sudo modprobe iptable_nat
sudo modprobe iptable_mangle
\end{lstlisting}

\section{Arquivos de Configuração e Principais Características}

\subsection{Arquivos de Configuração}

\begin{itemize}
    \item \textbf{/etc/iptables/rules.v4}: arquivo principal de regras IPv4
    \item \textbf{/etc/iptables/rules.v6}: arquivo principal de regras IPv6
    \item \textbf{/etc/default/iptables}: configurações padrão
    \item \textbf{/proc/net/ip\_tables\_names}: tabelas ativas
\end{itemize}

\subsection{Comandos Principais}

\begin{lstlisting}[language=bash, caption=Comandos básicos do iptables]
# listar todas as regras
sudo iptables -L -v -n

# listar regras com números de linha
sudo iptables -L -v -n --line-numbers

# limpar todas as regras
sudo iptables -F

# definir políticas padrão
sudo iptables -P INPUT DROP
sudo iptables -P OUTPUT ACCEPT
sudo iptables -P FORWARD DROP
\end{lstlisting}

\section{Glossário de Comandos e Opções Comuns}

Para construir e entender as regras do iptables, é fundamental conhecer o significado de suas principais opções (flags) e alvos (targets). Abaixo estão os mais utilizados neste guia.

\subsubsection{Opções de Gerenciamento de Regras}

\begin{itemize}
    \item \textbf{-A, --append}: Adiciona uma nova regra ao final de uma cadeia (chain).
    
    \textit{Exemplo}: \texttt{iptables -A INPUT ...}
    
    \item \textbf{-D, --delete}: Deleta uma regra de uma cadeia, seja pelo seu número ou pela sua especificação exata.
    
    \textit{Exemplo}: \texttt{iptables -D INPUT 1}
    
    \item \textbf{-P, --policy}: Define a política padrão para uma cadeia. Esta é a ação a ser tomada caso nenhum pacote corresponda a uma regra específica.
    
    \textit{Exemplo}: \texttt{iptables -P INPUT DROP}
    
    \item \textbf{-L, --list}: Lista todas as regras de uma cadeia.
    
    \item \textbf{-F, --flush}: Limpa (deleta) todas as regras de uma cadeia.
    
    \item \textbf{-X, --delete-chain}: Deleta uma cadeia personalizada que esteja vazia.
    
    \item \textbf{-C, --check}: Verifica se uma regra existe, sem aplicá-la.
\end{itemize}

\subsubsection{Parâmetros de Correspondência (Matching)}

\begin{itemize}
    \item \textbf{-p, --protocol}: Especifica o protocolo do pacote (ex: tcp, udp, icmp).
    
    \item \textbf{--dport, --destination-port}: Especifica a porta de destino para protocolos como TCP e UDP.
    
    \item \textbf{-i, --in-interface}: Especifica a interface de entrada pela qual um pacote foi recebido (ex: eth0, lo).
    
    \item \textbf{-o, --out-interface}: Especifica a interface de saída pela qual um pacote será enviado.
    
    \item \textbf{-m, --match}: Carrega um módulo de correspondência estendido. Essencial para regras mais complexas.
    
    \textit{Exemplo}: \texttt{-m state} para usar o módulo de estado da conexão.
    
    \item \textbf{--state}: Usado com o módulo \texttt{-m state}, verifica o estado de uma conexão (NEW, ESTABLISHED, RELATED).
    
    \item \textbf{--ctstate}: Usado com o módulo \texttt{-m conntrack}, é a forma moderna de verificar o estado da conexão.
\end{itemize}

\subsubsection{Alvos (Targets)}

O alvo é especificado pela opção \textbf{-j, --jump} e define o que fazer com um pacote que corresponde a uma regra.

\begin{itemize}
    \item \textbf{ACCEPT}: Aceita o pacote, permitindo que ele continue seu fluxo normal.
    
    \item \textbf{DROP}: Descarta o pacote silenciosamente. Nenhuma resposta é enviada à origem, o que é ideal para segurança, pois não revela a presença do firewall.
    
    \item \textbf{REJECT}: Rejeita o pacote, bloqueando-o, mas enviando uma mensagem de erro à origem (ex: "port unreachable").
    
    \item \textbf{LOG}: Registra o pacote nos logs do sistema (geralmente em \texttt{/var/log/syslog}). É um alvo que não interrompe o fluxo; após o log, o pacote continua para a próxima regra na cadeia.
    
    \item \textbf{MASQUERADE}: Usado na tabela nat, "camufla" os endereços de IP de uma rede interna, fazendo com que todos pareçam vir do endereço IP do firewall. Essencial para compartilhar uma conexão com a internet.
    
    \item \textbf{DNAT}: (Destination NAT) Altera o endereço de IP de destino do pacote. Usado para redirecionar portas (port forwarding).
\end{itemize}

\subsection{Exemplo de Configuração Básica}

\textbf{NOTA:} Para executar este script, primeiro salve-o em um arquivo (por exemplo, \texttt{firewall-rules.sh}) e depois conceda permissão de execução com o comando: \texttt{chmod +x firewall-rules.sh}. Em seguida, execute-o com \texttt{sudo ./firewall-rules.sh}.

\begin{lstlisting}[language=bash, caption=Configuração Básica de firewall]
#!/bin/bash

# limpar todas as regras
iptables -F
iptables -X
iptables -t nat -F
iptables -t nat -X
iptables -t mangle -F
iptables -t mangle -X

# definir políticas padrão
iptables -P INPUT DROP
iptables -P FORWARD DROP
iptables -P OUTPUT ACCEPT

# permitir tráfego local
iptables -A INPUT -i lo -j ACCEPT
iptables -A OUTPUT -o lo -j ACCEPT

# permitir conexões estabelecidas e relacionadas (essencial para o retorno de pacotes)
# ESTABLISHED: pacotes que fazem parte de uma conexão existente
# RELATED: pacotes de novas conexões que estão relacionadas a uma existente (ex: FTP)
iptables -A INPUT -m state --state ESTABLISHED,RELATED -j ACCEPT

# permitir SSH (porta 22)
iptables -A INPUT -p tcp --dport 22 -j ACCEPT

# permitir HTTP (porta 80)
iptables -A INPUT -p tcp --dport 80 -j ACCEPT

# permitir HTTPS (porta 443)
iptables -A INPUT -p tcp --dport 443 -j ACCEPT

# permitir ping (ICMP)
iptables -A INPUT -p icmp -j ACCEPT
\end{lstlisting}

\subsection{Tornando as Regras Persistentes}

Após executar o script \texttt{firewall-rules.sh}, as regras são aplicadas imediatamente, mas apenas na memória do kernel. Isso significa que, se o servidor for reiniciado, \textbf{essas regras serão perdidas}.

Para garantir que o seu firewall seja reativado com o mesmo conjunto de regras a cada inicialização, você deve salvá-las de forma permanente. O pacote \texttt{iptables-persistent}, que foi instalado anteriormente, é a ferramenta correta para esta tarefa.

O processo é feito em duas etapas:
\begin{enumerate}
    \item \textbf{Aplicar as regras} com o script, como já foi feito: \texttt{sudo ./firewall-rules.sh}.
    \item \textbf{Salvar as regras ativas} no disco com o seguinte comando:
\end{enumerate}

\begin{lstlisting}[language=bash, caption=Salvando as regras para persistência]
# Este comando salva as regras atuais do IPv4 e IPv6
sudo netfilter-persistent save
\end{lstlisting}

Ao executar este comando, o serviço \texttt{netfilter-persistent} irá ler as regras que estão ativas na memória e escrevê-las no arquivo \texttt{/etc/iptables/rules.v4}. A partir de agora, a cada boot do sistema, este serviço irá carregar automaticamente as regras salvas neste arquivo, garantindo que seu servidor esteja sempre protegido.

\section{Outras Opções, Descritivas de Segurança, Boas Práticas e Exemplos}

\subsection{Configuração de NAT}

\begin{lstlisting}[language=bash, caption=Configuração de NAT]
# habilitar NAT para rede interna
# nota: substitua 'eth0' pelo nome correto da sua interface de rede
# Use 'ip a' ou 'ifconfig' para identificar o nome da interface
iptables -t nat -A POSTROUTING -o eth0 -j MASQUERADE

# redirecionar porta externa para interna
iptables -t nat -A PREROUTING -p tcp --dport 8080 -j DNAT --to-dest 192.168.1.100:80
\end{lstlisting}

\textbf{NOTA sobre nomes de interfaces}: Em sistemas modernos, as interfaces de rede raramente usam nomes como \texttt{eth0}. Use comandos como \texttt{ip a} ou \texttt{ifconfig} para identificar o nome correto da interface em seu sistema (ex: \texttt{enp3s0}, \texttt{ens18}).

\subsection{Configuração de Logging}

\begin{lstlisting}[language=bash, caption=Configuração de logs]
# Log de tentativas de conexão SSH
iptables -A INPUT -p tcp --dport 22 -j LOG --log-prefix "SSH_ATTEMPT: "

# Log de pacotes rejeitados (útil para debugging)
iptables -A INPUT -j LOG --log-prefix "DROP: "
\end{lstlisting}

\subsection{Proteção Contra Ataques}

\begin{lstlisting}[language=bash, caption=Proteções de segurança]
# Proteção contra SYN flood
iptables -A INPUT -p tcp --syn -m limit --limit 1/s --limit-burst 3 -j ACCEPT

# Proteção contra port scanning
iptables -A INPUT -p tcp --tcp-flags ALL NONE -j DROP

# Proteção contra ataques de força bruta SSH (Versão corrigida)
# adicionar IPs que tentam novas conexões SSH a uma lista "recent"
iptables -A INPUT -p tcp --dport 22 -m conntrack --ctstate NEW -m recent --set --name SSH --rsource

# bloquear IPs na lista que fizerem mais de 4 tentativas em 60 segundos
iptables -A INPUT -p tcp --dport 22 -m conntrack --ctstate NEW -m recent --update --seconds 60 --hitcount 4 --name SSH --rsource -j DROP
\end{lstlisting}

\subsection{Debugging de Regras}

Uma das maiores dificuldades ao aprender iptables é descobrir por que um pacote está sendo bloqueado. Aqui estão técnicas úteis:

\begin{lstlisting}[language=bash, caption=Técnicas de debugging]
# adicionar logging antes da política DROP para diagnosticar regras
iptables -A INPUT -j LOG --log-prefix "PACOTE DESCARTADO: "

# verificar logs em tempo real
sudo tail -f /var/log/syslog | grep iptables

# Alternativa para sistemas com systemd-journald
# sudo journalctl -fu kernel | grep iptables

# testar regras específicas com contadores
iptables -A INPUT -p tcp --dport 80 -j ACCEPT
iptables -L INPUT -v -n  # ver contadores de pacotes (pkts) para confirmar que a regra foi atingida

# Usar iptables-save para ver todas as regras em formato legível
sudo iptables-save
\end{lstlisting}

\subsection{Boas Práticas}

\begin{enumerate}
    \item \textbf{Sempre testar regras em ambiente de desenvolvimento}
    \item \textbf{Manter backup das configurações atuais}
    \item \textbf{Documentar todas as regras criadas}
    \item \textbf{Monitorar logs regularmente}
    \item \textbf{Usar políticas restritivas por padrão}
    \item \textbf{Implementar rate limiting para serviços críticos}
    \item \textbf{Manter o sistema atualizado}
    \item \textbf{Usar \texttt{ss} em vez de \texttt{netstat} (obsoleto)}
\end{enumerate}

\subsection{Script de Backup e Restauração}

\begin{lstlisting}[language=bash, caption=Script de backup]
#!/bin/bash

# Certifique-se de que o diretório de backup existe
# mkdir -p /root/Backups_Firewall

# backup das regras atuais em um diretório seguro (/root/backups_firewall)
iptables-save > /root/backups_firewall/iptables_rules_$(date +%Y%m%d_%H%M%S).bak

# restaurar regras
# iptables-restore < /root/backups_firewall/iptables_rules_recentes.bak
\end{lstlisting}

\subsection{Monitoramento e Manutenção}

\begin{lstlisting}[language=bash, caption=Comandos de monitoramento]
# verificar estatísticas
sudo iptables -L -v -n

# monitorar logs em tempo real
sudo tail -f /var/log/syslog | grep iptables

# Alternativa para sistemas com systemd-journald
# sudo journalctl -fu kernel | grep iptables

# verificar conexões ativas (recomendado: usar ss em vez de netstat)
sudo ss -tuln

# verificar interfaces de rede
sudo ip a
\end{lstlisting}

\textbf{NOTA}: O comando \texttt{netstat} é considerado obsoleto. Use \texttt{ss} (socket statistics) que é mais eficiente é moderno. Em sistemas que usam systemd-journald, uma alternativa moderna para monitoramento de logs é o comando \texttt{journalctl -fu kernel}.

\section{Comandos Essenciais - Referência Rápida}

\begin{table}[H]
\centering
\begin{tabular}{@{}ll@{}}
\toprule
\textbf{Comando} & \textbf{Descrição} \\
\midrule
\texttt{iptables -L -v -n} & Listar todas as regras com estatísticas \\
\texttt{iptables -A INPUT -p tcp --dport 80 -j ACCEPT} & Adicionar regra para permitir HTTP \\
\texttt{iptables -D INPUT 1} & Deletar regra número 1 da cadeia INPUT \\
\texttt{iptables -F} & Limpar todas as regras \\
\texttt{iptables -P INPUT DROP} & Definir política padrão da cadeia INPUT \\
\texttt{iptables-save} & Salvar regras em arquivo \\
\texttt{iptables-restore < arquivo} & Restaurar regras de arquivo \\
\texttt{iptables -L -v -n --line-numbers} & Listar regras com números de linha \\
\texttt{iptables -C INPUT -p tcp --dport 22 -j ACCEPT} & Verificar se regra existe \\
\bottomrule
\end{tabular}
\caption{Comandos essenciais do iptables}
\end{table}

\section{Considerações Finais}

O netfilter/iptables é um sistema poderoso e essencial para a segurança de redes Linux. O netfilter fornece o framework de firewall no kernel, enquanto o iptables oferece uma interface flexível para gerenciar as regras. Juntos, permitem implementar políticas de segurança complexas, desde configurações básicas até setups avançados de alta disponibilidade.

Principais pontos a considerar:

\begin{itemize}
    \item \textbf{Complexidade}: A curva de aprendizado pode ser íngreme, mas o domínio do sistema é fundamental
    \item \textbf{Performance}: regras mal configuradas podem impactar o desempenho da rede
    \item \textbf{Manutenção}: configurações devem ser revisadas e atualizadas regularmente
    \item \textbf{Documentação}: manter documentação atualizada é crucial para troubleshooting
    \item \textbf{Testes}: sempre testar configurações em ambiente controlado antes da produção
    \item \textbf{Transição para nftables}: esteja ciente da transição em andamento para nftables
\end{itemize}

Para ambientes de produção, considere também:

\begin{itemize}
    \item implementar failover para alta disponibilidade
    \item usar ferramentas de monitoramento como Nagios ou Zabbix
    \item implementar alertas automáticos para tentativas de intrusão
    \item manter procedimentos de recuperação de desastres
    \item treinar equipe de suporte nas configurações implementadas
    \item considerar a migração para nftables em novos projetos
\end{itemize}

\section{Referências Bibliográficas}

\begin{thebibliography}{99}

\bibitem{debian-release-notes}
Debian Project. \textit{Debian 12 "Bookworm" Release Notes}. Disponível em: \url{https://www.debian.org/releases/bookworm/s390x/release-notes.pt-br.pdf}. Acesso em: 2024.

\bibitem{netfilter-documentation}
Netfilter Project. \textit{Netfilter Documentation}. Disponível em: \url{https://www.netfilter.org/documentation/}. Acesso em: 2024.

\bibitem{wikipedia-debian}
Wikipedia. \textit{Debian}. Disponível em: \url{https://pt.wikipedia.org/wiki/Debian}. Acesso em: 2024.

\bibitem{ovh-firewall}
OVHcloud. \textit{Firewall iptables}. Disponível em: \url{https://help.ovhcloud.com/csm/pt-dedicated-servers-firewall-iptables?id=kb_article_view&sysparm_article=KB0043443}. Acesso em: 2024.

\bibitem{hostinger-iptables}
Hostinger. \textit{Tutorial iptables}. Disponível em: \url{https://www.hostinger.com/br/tutoriais/tutorial-iptables}. Acesso em: 2024.

\bibitem{redhat-iptables}
Red Hat. \textit{iptables}. Disponível em: \url{https://www.redhat.com/en/blog/iptables}. Acesso em: 2024.

\end{thebibliography}

%inserindo anexos
%\appendix

%\section{Anexo I}
%O anexo bla .....

\end{document} %finaliza o documento
