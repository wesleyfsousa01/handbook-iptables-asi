\documentclass[12pt,fleqn]{article}
%\usepackage {psfig,epsfig} % para incluir figuras em PostScript
\usepackage{amsfonts,amsthm,amsopn,amssymb,latexsym}
\usepackage{graphicx}
\usepackage[T1]{fontenc}
\usepackage[brazil]{babel}
\usepackage{geometry}
\usepackage[latin1]{inputenc}
\usepackage[intlimits]{amsmath}
\usepackage{listings}
\usepackage{xcolor}
\usepackage{hyperref}

% Configuração para código
\lstset{
    basicstyle=\ttfamily\small,
    breaklines=true,
    frame=single,
    numbers=left,
    numberstyle=\tiny,
    keywordstyle=\color{blue},
    commentstyle=\color{green!60!black},
    stringstyle=\color{red},
    backgroundcolor=\color{gray!10},
    showstringspaces=false
}

%alguns macros
\newcommand{\R}{\ensuremath{\mathbb{R}}}
\newcommand{\Rn}{{\ensuremath{\mathbb{R}}}^{n}}
\newcommand{\Rm}{{\ensuremath{\mathbb{R}}}^{m}}
\newcommand{\Rmn}{{\ensuremath{\mathbb{R}}}^{{m}\times{n}}}
\newcommand{\contcaption}[1]{\vspace*{-0.6\baselineskip}\begin{center}#1\end{center}\vspace*{-0.6\baselineskip}}
%=======================================================================
% Dimensões da página
\usepackage{a4}                       % tamanho da página
\setlength{\textwidth}{16.0cm}        % largura do texto
\setlength{\textheight}{9.0in}        % tamanho do texto (sem head, etc)
\renewcommand{\baselinestretch}{1.15} % espaçamento entre linhas
\addtolength{\topmargin}{-1cm}        % espaço entre o head e a margem
\setlength{\oddsidemargin}{-0.1cm}    % espaço entre o texto e a margem
       
% Ser indulgente no preenchimento das linhas
\sloppy
 
\begin{document}

\pagestyle {empty}

% Páginas iniciais

\newpage

%\tableofcontents

% Numeração em romanos para páginas iniciais (sumários, listas, etc)
%\pagenumbering {roman}
\pagestyle {plain}

\setcounter{page}{0} \pagenumbering{arabic}

\setlength{\parindent}{0in}  %espaco entre paragrafo e margem 
% Espaçamento entre parágrafos
\parskip 5pt  

\center {\huge \textbf{Capítulo 7 - Servidor Firewall (iptables) e Aplicações}}
\center {\textit{Autor: Grupo ASI - IFG Câmpus Formosa}}

\section {Ficha Técnica}  

\begin{tabular}{|c|c|}
	\hline
	\textbf{Serviço} & \textbf{Descrição}\\\hline 
	Método de Comunicação & Protocolo TCP/UDP, portas configuráveis (22, 80, 443, etc.) \\\hline
	Funções & Controle de tráfego de rede, filtragem de pacotes, NAT, balanceamento de carga\\\hline
	Pacote para instalação & iptables (já incluído no kernel Linux), iptables-persistent\\\hline
	Script de controle & /etc/init.d/netfilter-persistent, systemctl status iptables\\\hline
\end{tabular}

\section{Descrição do Servidor}

O iptables é um firewall baseado em kernel do Linux que permite controlar o tráfego de rede através de regras de filtragem de pacotes. É uma ferramenta fundamental para segurança de rede, permitindo:

\begin{itemize}
    \item \textbf{Filtragem de Pacotes}: Controlar quais pacotes podem entrar ou sair do sistema
    \item \textbf{NAT (Network Address Translation)}: Tradução de endereços de rede
    \item \textbf{Mangle}: Modificação de cabeçalhos de pacotes
    \item \textbf{Logging}: Registro de atividades de rede para auditoria
\end{itemize}

O iptables trabalha com três tabelas principais:
\begin{itemize}
    \item \textbf{filter}: Tabela padrão para filtragem de pacotes
    \item \textbf{nat}: Para tradução de endereços de rede
    \item \textbf{mangle}: Para modificação de pacotes
\end{itemize}

E cinco cadeias (chains) principais:
\begin{itemize}
    \item \textbf{INPUT}: Pacotes destinados ao sistema local
    \item \textbf{OUTPUT}: Pacotes originados do sistema local
    \item \textbf{FORWARD}: Pacotes que passam pelo sistema (roteamento)
    \item \textbf{PREROUTING}: Pacotes que chegam (NAT)
    \item \textbf{POSTROUTING}: Pacotes que saem (NAT)
\end{itemize}

\section{Instalação}

O iptables já vem incluído no kernel Linux, mas para uma configuração completa, é necessário instalar alguns pacotes adicionais:

\begin{lstlisting}[language=bash, caption=Instalação do iptables]
# Atualizar repositórios
sudo apt update

# Instalar iptables e ferramentas relacionadas
sudo apt install iptables iptables-persistent netfilter-persistent

# Verificar se o iptables está funcionando
sudo iptables -L -v
\end{lstlisting}

Para verificar se o módulo do kernel está carregado:

\begin{lstlisting}[language=bash, caption=Verificação de módulos]
# Verificar módulos carregados
lsmod | grep iptable

# Carregar módulos se necessário
sudo modprobe iptable_filter
sudo modprobe iptable_nat
sudo modprobe iptable_mangle
\end{lstlisting}

\section {Arquivos de Configuração e Principais Características}

\subsection{Arquivos de Configuração}

\begin{itemize}
    \item \textbf{/etc/iptables/rules.v4}: Arquivo principal de regras IPv4
    \item \textbf{/etc/iptables/rules.v6}: Arquivo principal de regras IPv6
    \item \textbf{/etc/default/iptables}: Configurações padrão
    \item \textbf{/proc/net/ip\_tables\_names}: Tabelas ativas
\end{itemize}

\subsection{Comandos Principais}

\begin{lstlisting}[language=bash, caption=Comandos básicos do iptables]
# Listar todas as regras
sudo iptables -L -v -n

# Listar regras com números de linha
sudo iptables -L -v -n --line-numbers

# Limpar todas as regras
sudo iptables -F

# Definir políticas padrão
sudo iptables -P INPUT DROP
sudo iptables -P OUTPUT ACCEPT
sudo iptables -P FORWARD DROP
\end{lstlisting}

\subsection{Exemplo de Configuração Básica}

\begin{lstlisting}[language=bash, caption=Configuração básica de firewall]
#!/bin/bash

# Limpar todas as regras
iptables -F
iptables -X
iptables -t nat -F
iptables -t nat -X
iptables -t mangle -F
iptables -t mangle -X

# Definir políticas padrão
iptables -P INPUT DROP
iptables -P FORWARD DROP
iptables -P OUTPUT ACCEPT

# Permitir tráfego local
iptables -A INPUT -i lo -j ACCEPT
iptables -A OUTPUT -o lo -j ACCEPT

# Permitir conexões estabelecidas
iptables -A INPUT -m state --state ESTABLISHED,RELATED -j ACCEPT

# Permitir SSH (porta 22)
iptables -A INPUT -p tcp --dport 22 -j ACCEPT

# Permitir HTTP (porta 80)
iptables -A INPUT -p tcp --dport 80 -j ACCEPT

# Permitir HTTPS (porta 443)
iptables -A INPUT -p tcp --dport 443 -j ACCEPT

# Permitir ping (ICMP)
iptables -A INPUT -p icmp -j ACCEPT
\end{lstlisting}

\section {Outras opções/Descritivas de segurança/Boas práticas/Exemplos}

\subsection{Configuração de NAT}

\begin{lstlisting}[language=bash, caption=Configuração de NAT]
# Habilitar NAT para rede interna
iptables -t nat -A POSTROUTING -o eth0 -j MASQUERADE

# Redirecionar porta externa para interna
iptables -t nat -A PREROUTING -p tcp --dport 8080 -j DNAT --to-dest 192.168.1.100:80
\end{lstlisting}

\subsection{Configuração de Logging}

\begin{lstlisting}[language=bash, caption=Configuração de logs]
# Log de tentativas de conexão SSH
iptables -A INPUT -p tcp --dport 22 -j LOG --log-prefix "SSH_ATTEMPT: "

# Log de pacotes rejeitados
iptables -A INPUT -j LOG --log-prefix "DROP: "
\end{lstlisting}

\subsection{Proteção contra Ataques}

\begin{lstlisting}[language=bash, caption=Proteções de segurança]
# Proteção contra SYN flood
iptables -A INPUT -p tcp --syn -m limit --limit 1/s --limit-burst 3 -j ACCEPT

# Proteção contra port scanning
iptables -A INPUT -p tcp --tcp-flags ALL NONE -j DROP

# Proteção contra ataques de força bruta SSH
iptables -A INPUT -p tcp --dport 22 -m recent --name SSH --set
iptables -A INPUT -p tcp --dport 22 -m recent --name SSH --update --seconds 60 --hitcount 4 -j DROP
\end{lstlisting}

\subsection{Boas Práticas}

\begin{enumerate}
    \item \textbf{Sempre testar regras em ambiente de desenvolvimento}
    \item \textbf{Manter backup das configurações atuais}
    \item \textbf{Documentar todas as regras criadas}
    \item \textbf{Monitorar logs regularmente}
    \item \textbf{Usar políticas restritivas por padrão}
    \item \textbf{Implementar rate limiting para serviços críticos}
    \item \textbf{Manter o sistema atualizado}
\end{enumerate}

\subsection{Script de Backup e Restauração}

\begin{lstlisting}[language=bash, caption=Script de backup]
#!/bin/bash

# Backup das regras atuais
iptables-save > /backup/iptables_rules_$(date +%Y%m%d_%H%M%S).bak

# Restaurar regras
# iptables-restore < /backup/iptables_rules_20231201_143022.bak
\end{lstlisting}

\subsection{Monitoramento e Manutenção}

\begin{lstlisting}[language=bash, caption=Comandos de monitoramento]
# Verificar estatísticas
sudo iptables -L -v -n

# Monitorar logs em tempo real
sudo tail -f /var/log/syslog | grep iptables

# Verificar conexões ativas
sudo netstat -tuln

# Verificar processos de rede
sudo ss -tuln
\end{lstlisting}

\section{Considerações finais}

O iptables é uma ferramenta poderosa e essencial para a segurança de redes Linux. Sua flexibilidade permite implementar políticas de segurança complexas, desde configurações básicas até setups avançados de alta disponibilidade.

Principais pontos a considerar:

\begin{itemize}
    \item \textbf{Complexidade}: A curva de aprendizado pode ser íngreme, mas o domínio da ferramenta é fundamental
    \item \textbf{Performance}: Regras mal configuradas podem impactar o desempenho da rede
    \item \textbf{Manutenção}: Configurações devem ser revisadas e atualizadas regularmente
    \item \textbf{Documentação}: Manter documentação atualizada é crucial para troubleshooting
    \item \textbf{Testes}: Sempre testar configurações em ambiente controlado antes da produção
\end{itemize}

Para ambientes de produção, considere também:

\begin{itemize}
    \item Implementar failover para alta disponibilidade
    \item Usar ferramentas de monitoramento como Nagios ou Zabbix
    \item Implementar alertas automáticos para tentativas de intrusão
    \item Manter procedimentos de recuperação de desastres
    \item Treinar equipe de suporte nas configurações implementadas
\end{itemize}

%Incluindo referências bibliográficas
%\bibliographystyle{plain} %define o estilo         
%\bibliography{bibliografia} %busca o arquivo

%inserindo anexos
%\appendix

%\section{Anexo I}
%O anexo bla .....

\end{document} %finaliza o documento
